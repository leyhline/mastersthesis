\chapter{Foundations}\label{chap:foundations}

Before diving into the \emph{Cast Calculus} we introduce the foundation, the \emph{Label Dependent Lambda Calculus} (LDLC) and its implementation LDGV.

\section{Label Dependent Lambda Calculus}\label{sec:ldlc}

The LDLC is the foundation of this work. Its main aspect is the gain of additional simplicity in context of session types. The introduction of labels results in decoupled operations when sending and receiving functions, hence being more lightweight regarding its semantics. See \cite{thiemann2019}. However, session types are not of special relevance when introducing casts. For brevity, we will omit the handling of session types, focusing on labels instead.

A list of language constructs is found in figures~\ref{fig:ldlc-expressions}~to~\ref{fig:ldlc-values}. There is support for common functionality like variables, expressions and types,\footnote{
Variable names for expressions have to start with a lowercase letter, e.g. \texttt{\valb x = ()}; for types its an uppercase letter: \texttt{\typeb X : Unit}.} functions and pairs as well as numerical operations and values. In general---as implied by its name---it is based on \emph{Simply Typed Lambda Calculus} (details can be found in most introductory literature, e.g. \cite{pierce2002}). As in many functional languages, there is also the unit type $\Unit$ encompassing its only value ``$()$''. 

Another two language concepts are more interesting: First and foremost, the support for labels $\ell$. Labels can have arbitrary names. A mathematical set of distinct labels forms a type. Moreover, there is a recursor over natural numbers $\rec V~M~x.f.N$. This allows for recursively constructing expressions by using base expression $M$ for $n=0$ and otherwise expression $N$ with appropriate substitutions for $x$ and $f$.

\begin{figure}
\begin{align*}
 M,N,P ::=
    & \enspace x \enspace|\enspace
      () \enspace|\enspace
      \ell \enspace|\enspace
      x \in \mathbb{R} \\
    & |\enspace M \circ N & \circ \in \{+,-,\times,\div\} \\
    & |\enspace \case M \{\overline{\ell : N_\ell}^{\ell \in L} \} & \text{label expression matching} \\
    & |\enspace \rec V~M~x.f.N & \mathbb{N}\text{ recursor} \\
    & |\enspace \lambda (x : A).M & \text{abstraction} \\
    & |\enspace M~N & \text{application}\\
    & |\enspace \letb x = M \inb N & \text{bind in $N$}\\
    & |\enspace \letb \langle x, y \rangle = M \inb N & \text{bind pair in $N$} \\
    & |\enspace \langle x = M, N \rangle & \text{pair construction}\\
    & |\enspace \fstb M \enspace|\enspace \sndb M & \text{pair elimination}
\end{align*}
\caption{Expressions in LDLC}
\label{fig:ldlc-expressions}
\end{figure}

\begin{figure}
\begin{align*}
 A, B ::=
    & \enspace t \enspace|\enspace
      \Unit
    \\ &  |\enspace \Int \enspace|\enspace
      \Nat \enspace|\enspace
      \Double \\
    & |\enspace \{ \ell_1, \dots, \ell_n \} & \text{finite label set} \\
    & |\enspace \Sigma(x:A)B & \text{pair type} \\
    & |\enspace \Pi(x:A)B & \text{function type} \\
    & |\enspace \case M \{\overline{\ell : A_\ell}^{\ell \in L} \} & \text{label type matching} \\
    & |\enspace \rec V~A~t.B & \text{recursor type}
\end{align*}
\caption{Types in LDLC}
\label{fig:ldlc-types}
\end{figure}

\begin{figure}
\begin{align*}
    V, W ::=
    & \enspace () \enspace|\enspace
      \ell \enspace|\enspace
      x \in \mathbb{R} \enspace
     |\enspace \langle V, W \rangle
\end{align*}
\caption{Values in LDLC}
\label{fig:ldlc-values}
\end{figure}

\subsection{Labels and Label Sets}

The main feature of LDLC is its use of labels. A label evaluates to itself, thus it forms its own value. But what about its type? Any finite non-empty set of labels $L$ is a type. For instance, this is sufficient to declare a boolean type as seen in equation~\ref{eqn:bool}.

The minimal type is the singleton type of its corresponding label. For label $\ell$ this is $\{\ell\}$. Apart from that there might be a bottom type $\bot$ which corresponds to a hypothetical empty label set. Neither type $\bot$ nor the empty label set are valid types in LDLC.

Predominantly, labels appear in context of \case terms. These may occur as expressions or as types and act as eliminators for labels. They provide a mapping from each label of the corresponding label set to an expression or a type. Hence the name of the calculus, \emph{Label Dependent} Lambda Calculus. The type of an expression may depend on a label value.

\subsection{Recursor over Natural Numbers}\label{sec:ldgv-recursor}

In addition to label dependency, there is another construct used for modeling dependency on natural numbers: $\rec V~M~x.f.N$

Value $V$ must be a natural number value which can also be written recursively using the constructors $Z$ (for zero) and $S(V)$ for the successor of $V$. An encoding for a natural number is given by $\overline{n} = S(\dots S(Z))$. When evaluating a recursor, there is a case differentiation by value $V$.

If $V=Z$ the whole expression is reduced to $M$, discarding its latter term. Otherwise, for $V=S(V')$, the \rec expression is reduced to expression $N$ with the following two substitutions: variable $x$ in $N$ is substituted by $V'$, the predecessor of $S(V')$; variable $f$ by $\rec V'~M~x.f.N$, the result of the ``previous'' evaluation of the recursor.

\begin{figure}
 \begin{align*}
 \rec Z~&M~x.f.N \longrightarrow M \\
 \rec S(V)~&M~x.f.N \longrightarrow N [V/x] [\rec V~M~x.f.N/f] \\
 \rec Z~&A~t.B \longrightarrow A \\
 \rec S(V)~&A~t.B \longrightarrow B [\rec V~A~t.B/t]
\end{align*}
\caption{Nat. recursor expression reduction and type reduction}
\label{fig:ldlc-rec-exp}
\end{figure}


On the type level it is similar when evaluating a recursor type $\rec V~A~t.B$. For $V=Z$, the resulting type is simply $A$. As for $V=S(V')$, the resulting type is given by $B$ after substituting type identifier $t$ in $B$ with $\rec V'~A~t.B$. Both reductions are concisely listed in figure~\ref{fig:ldlc-rec-exp}.

\subsection{Subtyping}

Without subtyping, a type system lacks flexibility. Assuming a lambda expression $\lambda ( x: \Int ).M$, what will happen when applying an expression of type \textbf{Nat}? Obviously, there is a difference between natural numbers and integers. But at the same time, every natural number is also an integer. ($\mathbb{N} \subset \mathbb{Z}$) By formulating consistent rules---not necessarily involving subset relations---a type system will accept a wider range of types. Especially in object-oriented languages supporting inheritance this is practically a necessary feature.\footnote{A common example is class \emph{Dog} inheriting from class \emph{Animal}; then methods accepting arguments of type \emph{Animal} will also accept type \emph{Dog}.} Subtyping for dependent types entails some peculiarities that were  studied in various papers during the last three decades, e.g. by \cite{aspinall2001}.

In figure~\ref{fig:ldgv-subtyping} we give an excerpt of relevant subtyping rules of LDLC. Handling of function types (\textsc{A-Sub-Pi}) and pair types (\textsc{A-Sub-Sigma}) is of special relevance since they are composite types, thus they need specification on how to recursively apply subtyping to their component types. For CCLDLC these rules gain additional relevance since they are necessary not only for type checking but now for type casts during runtime, too.

Since a label type is a set of labels we can use set comparisons for subtyping. If one label type $L$ is a subset of another label type $L'$, then $L$ is a subtype of $L'$. Thus, there is no $L$ that is a subtype of the empty set $\emptyset$ and equivalently of bottom type $\bot$. Subtyping in LDLC is closed under transitivity.

\begin{figure}
\begin{mathpar}
\inferrule[A-Sub-Unit]
  {\Gamma \Vdash \Unit \leq \Unit}{}
\and
\inferrule[A-Sub-Label]
  {L \subseteq L'}{\Gamma \Vdash L \leq L'}
\\
\inferrule[A-Sub-Pi]
  {\Gamma \Vdash A' \leq A \and \Gamma,x:A' \Vdash B \leq B'}{\Gamma \Vdash \Pi(x:A)B \leq \Pi(x:A')B'}
\\
\inferrule[A-Sub-Sigma]
  {\Gamma \Vdash A \leq A' \and \Gamma, x:A \Vdash B \leq B'}
  {\Gamma \Vdash \Sigma(x:A)B \leq \Sigma(x:A')B'}
\\
\Gamma ::= \cdot \enspace | \enspace \Gamma, x:A
\end{mathpar}
\caption[Subtyping rules in LDGV (excerpt)]{Subtyping rules in LDGV (excerpt) w.r.t. type environment $\Gamma$}
\label{fig:ldgv-subtyping}
\end{figure}


\section{LDGV Interpreter}

LDGV stands for ``Label Dependency'' and ``Gay \& Vasconcelos'', the authors of multiple papers about Session Types, e.g. \cite{gayvasconcelos2010}. In this work's context, it is the name of an interpreter for LDLC and its extension CCLDLC. It is written in the Haskell programming language.

Its syntax is similar the calculus from the previous section. We give an excerpt in listing~\ref{lst:ldgv-expressions}. In general, keywords and structure are identical save for labels. To uniquely identify and tokenize a label it needs to be a combination of alphanumeric characters, prefixed by an apostrophe. Furthermore, as opposed to the underlying LDLC, there is basic string support. A combination of characters enclosed by quotation marks \texttt{"} is read as string expression.

\begin{lstlisting}[float,language=ldgv,
  caption={Expressions in LDGV (excerpt) with expressions \texttt{M,N,P}, variable \texttt{x} and type \texttt{A}},
  label=lst:ldgv-expressions]
()          -- unit
M + N       -- and other arithmetic
10          -- numbers
"foo bar"   -- strings
x           -- variables
'foo        -- labels
case M {'foo: M, 'bar: P}
fn(x: A) M  -- lambda function
<x = M, N>  -- pair
\end{lstlisting}

The syntax for types is given in listing~\ref{lst:ldgv-types}. Writing out function and pair types differs from the calculus' notation. Function types of multiple arguments are built by recursion, e.g. \texttt{(x:A) -> (y:B) -> C}.

\begin{lstlisting}[float,language=ldgv,
  caption={Types in LDGV (excerpt) with expression \texttt{M}, variable \texttt{x}, type identifier \texttt{T} and types \texttt{A,B}},
  label=lst:ldgv-types]
Unit
Int
Nat
Double
String
T             -- type identifier
{'foo, 'bar}  -- label type
case M {'foo: A, 'bar: B}
(x: A) -> B   -- function type
[x: A, B]     -- pair type
\end{lstlisting}

For an actual program there need to be statements, especially declarations. They create the global environment, binding expressions to variables and types to type identifiers\footnote{For type identifiers, the programmer has to supply a \emph{kind}. For brevity, we omit the introduction of kinds, generally assuming kind \texttt{\textasciitilde un} in declarations.}. When running the interpreter on a given program, it outputs the evaluation result of the expression bound to variable \texttt{main}. Furthermore, there are type assertions. The full list of statements in given in listing~\ref{lst:ldgv-statements}.

\begin{lstlisting}[float,language=ldgv,
  caption={Statements in LDGV, with expression \texttt{M} and types \texttt{A,B}},
  label=lst:ldgv-statements]
val x = M         -- variable declaration
val x : A         -- variable signature declaration
type T : ~un = A  -- type declaration
A <: B            -- subtype assertion
A =: B            -- type equivalence assertion
\end{lstlisting}

Translating equation~\ref{eqn:ldlc} into a valid LDGV program results in the sample program in figure~\ref{lst:ldgv-program}. The variable's type signatures \texttt{val x:A} are optional.
Variable \texttt{main} gives an example application, fully evaluating to value \texttt{'F} (representing boolean \emph{False}).

\begin{lstlisting}[float,language=ldgv,numbers=left,
  caption=Sample LDGV program,
  label=lst:ldgv-program]
type Bool : ~un = {'T, 'F}

val not : (a: Bool) -> Bool
val not   (a: Bool) = (case a {'T: 'F, 'F: 'T})

val f : (x: Bool)
         -> (y: case x {'T: Int, 'F: Bool})
             -> case x {'T: Int, 'F: Bool}
val f = fn(x: Bool)
          fn(y: case x {'T: Int, 'F: Bool})
            case x {'T: 17+y, 'F: not y}

val main = f 'F 'T
\end{lstlisting}
