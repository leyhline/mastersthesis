\chapter{Program Structure}\label{chap:Program Structure}

LDGV is built out of a number of modules. There is a clear hierarchy as seen in the dependency graph in figure~\ref{fig:dependency-graph}. Each functionality builds an independent group of modules:

\begin{itemize}
 \item Module \texttt{Parsing}, including tokenization
 \item Module \texttt{Typechecker}, including modules for typechecking expressions and another one for subtyping in particular.
 \item Module \texttt{Interpreter} with data types for environment and values.
 \item Module \texttt{C.Generate}, the code generator for the C backend
\end{itemize}

Most modules depend on the \texttt{Syntax} module. It provides data types for language constructs like Expressions, Types and Literals. When invoking the program's functionality, the relevant module functions are invoked in order, e.g. when running the interpreter it (1) parses the source code, (2) runs the type checker on the AST and on success, it finally (3) interprets the AST's \texttt{main} function if available and returns its value.

\begin{figure}
 \centering
 \includegraphics[width=20cm,angle=90]{figures/depgraph_new_concept.dot.pdf}
 % depgraph_new_concept.dot.pdf: 1197x332 px, 72dpi, 42.23x11.71 cm, bb=0 0 1197 332
 \caption{Module dependencies of LDGV, slightly simplified without utility modules.}
 \label{fig:dependency-graph}
\end{figure}

Since the module groups are seperated, only the Interpreter knows about values, though there is no dependency on the type checker. The interpreter for LDLC does not need to know about types, its only concern is the evaluation of expressions, i.e. after type checking, \emph{type erasure} occurs, simplifying the implementation (see: \cite{crary2002}).

\section{CCLDLC Adaptions}

For the cast calculus, most adaptions happen in the interpreter, implementing runtime casts. Parts of the aforementioned simplicity has to be abandoned to introduce type passing when necessary.

But before, we need to implement preliminary language constructs, starting with the most basic module: In the \texttt{Syntax} module, we add a constructor for the cast $(M : A \Rightarrow B)$ to the data type for expressions, accepting an expression $M$ and two types $A$ and $B$. We also add two argumentless types: bottom type $\bot$ and dynamic type $\star$. In section~\ref{sec:cc-expressions} we also mentioned the top type $\top$, which we do not implement. This is because there is no expression of type $\top$. When declaring a function taking an argument of arbitary type, the programmer will use the dynamic type $\lambda(x:\star)M$. This ensures the handling of variable $x$ at runtime. A function $\lambda(x:\top)M$ is only sound if $M = \blame$, which we do not allow because of its unnecessary complexity.

The modules connected to \texttt{Parsing} need to reflect these modifications. We introduce tokens for the dynamic type and the bottom type and extend the grammar accordingly. We also add an expression for casts. See figure~\ref{fig:ccldlc-grammar}.

The module group forming the \texttt{Typechecker} needs to handle the new types. $\bot$ is made a subtype to everything and the subtyping rules for $\star$ are implemented without introducing transitivity. Finally, we check that for an expression $(M : A \Rightarrow B)$ the encapsulated expression $M$ is of type $A$ while the cast itself is handled at runtime by the interpreter.

The interpreter (\todo{and compiler?}) needs most adjustments (there will be a detailed description in the next chapter?) \dots

\todo{What about the C backend? Do this later!}

\begin{figure}
\begin{lstlisting}[language=ldgv]
Bot           -- bottom type
*             -- dynamic type
exp : A => B  -- cast expression
\end{lstlisting}
\caption{Extensions to the parser for CCLDLC}
\label{fig:ccldlc-grammar}
\end{figure}
