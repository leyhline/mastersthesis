\chapter{Implementation}\label{chap:implementation}

The actual implementation uses the Haskell programming language and related tools. Adjustments in all four module groups as defined in chapter~\ref{chap:program-structure} are necessary. Furthermore, the data types in the fundamental \texttt{Syntax} module must be extended with the additional expression and types introduced in section~\ref{sec:cc-expressions}.

\begin{lstlisting}[language=Haskell]
data Exp = ...
  | Cast Exp Type Type  -- M : A => B

data Type = ...
  | TBot  -- bottom type
  | TDyn  -- dynamic type *
\end{lstlisting}

\section{Parsing}

As usual, there is a tokenization step before the actual parsing, turning a string into tokens. For this purpose, we use the tool \emph{Alex: A lexical analyser generator for Haskell}\footnote{\url{https://www.haskell.org/alex/}}, maintained by Simon Marlow. It nicely integrates with Haskell. We add but a single token, namely the double arrow used in cast expressions:

\begin{lstlisting}[language=Haskell]
data Token = ... | DoubleArrow | ...
"=>"  { tok $ const DoubleArrow }
\end{lstlisting}

For parsing, we use \emph{Happy: The Parser Generator for Haskell}\footnote{\url{https://www.haskell.org/happy/}}, also maintained by Simon Marlow. Parsing is slightly more complicated since we need to add the dynamic type $\star$ and the cast expression $M : A \Rightarrow B$.

\begin{lstlisting}[language=Haskell]
%token
  ...
  '=>'  { T _ T.DoubleArrow }
  '*'   { T _ (T.Sym '*') }

Exp = ...
  | Exp ':' Typ '=>' Typ  { Cast $1 $3 $5 }

Typ = ATyp | ...

ATyp = ...
  | '*'  { TDyn }
\end{lstlisting}
